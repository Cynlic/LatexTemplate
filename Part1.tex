\hypertarget{introduction}{%
\section{Introduction}\label{introduction}}

The discipline of ethnomusicology has, since its founding articulations
as a distinct field of inquiry, been delicately positioned between the
twin poles of musical performance and anthropology. Indeed, since the
publication of Mantle Hood's 1960 essay ``the Challenge of
Bi-Musicality'' and Allan Merriam's equally influential publication,
\emph{The Anthropology of Music} in 1964, ethnomusicology as a
discipline has maintained a vexed relationship to artistic practice,
simultaneously the site of tremendous intellectual and ethnographic
effort and the least representational mode of the discipline's academic
output. Buoyed in part by a destabilization of the authoritative
ethnographic text during ``Writing Culture debates'' which dominated the
anthropology of the late 1980s (see: Zenker 2014), ethnomusicology has
asserted the validity of musicking itself as a \emph{constitutive
element} in a research process which makes claims on the matrices of
social and cultural practice. This desire to explain the ``special
ontology of musical being,'' which bound the non-representational
qualities of research to long-standing methods in ethnographic practice,
formed in parallel to a resurgent interest in ``the sensory'' as
articulated in anthropological discourse, particularly among those
anthropologists with previous experience in the arts (Titon 2008: 32;
Pink: 2012; Ingold and Gatt 2013). In explaining this development Karen
Nakamura has distinguished two particular modes of engagement in the
turn to what has been called ``sensory ethnography'': the
``aesthetic-sensual'' dimension and the ``multisensory-experimental''
one (2013: 133). Taking a cue from Deborah Wong and Stephen Slawek
(2008; 1994), I here contend that ethnomusicological performance and
other forms of what I term (after Haraway) ``situated art'' should
placed in the context of this aesthetic-sensual realm of sensory
ethnography; further I maintain that these modes of engagement have a
unique value in the antagonistic relationship they pose as aesthetic
objects to epistemic strictures.

\hypertarget{ethnomusicology-and-performance-practice}{%
\section{Ethnomusicology and Performance
Practice}\label{ethnomusicology-and-performance-practice}}

In developing a theory of ethnomusicological performance practice, it is
necessary to briefly revisit how such notions are constructed in
relation what may be called the formative object of musical study. In
``The Scope, Aim, and Method of Musicology,'' an 1883 document which has
come to crystalize the historical aims of musical study, historian Guido
Adler laid out what he saw as basic properties of the discipline,
foregrounding in particular an inductive method which would cast music
history as a kind of empirical science. Adler's writing on this topic is
instructive insofar as it cites the tight interrelation of an art and
its ``science'' as a key by-product of induction:

\begin{quote}
To attain his main task, namely, the research of the laws of art of
diverse periods and their organic combination and development, the
historian of art utilises the same methodology as that of the
investigator of nature; that is, by preference the inductive
method\ldots The setting up of the highest laws of art and their
practical utilisation in musical paedagogics reveal the science in
unmediated contact with the actual life of art. The science attains its
goal to its fullest extent only when it remains in living contact with
art. Art and the science of art do not exist in separate compartments,
the boundaries of which are sharply drawn; rather it is far more one and
the same field, and only the way in which each is treated differs.
(Mugglestone 1983: 16).
\end{quote}

This insistent emphasis on the utility of scientific knowledge in
relation to the further production of art is paramount to Adler's
unifying vision, itself stated in the essay's conclusion as ``the
Discovery of the True and the Advancement of the Beautiful'' (18).
Tellingly, Adler assigns the nascent investigations of comparative
musicology, the discipline from which ethnomusicology eventually
developed, to his ``systematic'' subdivision of the musicological
project, grouped with music theory and other adjacent pursuits (13).
Thus the roots of ethnomusicology, which itself can be credibly seen as
a socio-cultural critique of the comparative project, are still
fundamentally tied to the relation of knowledge to the performance of
music. Such a notion might be obvious on its face --that music making
and its study are bound together in some capacity should not
particularly surprising -- but what I find necessary to highlight here
is the extent to which nineteenth century notions about the production
of knowledge through a scientific (or here we might say ``positivistic''
in historical parlance) practice form an over-arching discourse from
which various subfields may depart.

This intricate theoretical dance between art and its science, always in
the midst of revealing novel truths and resultant beauties, becomes
quite complex when shifting to the 1950s and 1960s, a period of
substantive concretization in the ethnomusicological discipline. Though
the differences between the historical and ethnological strands of
musicology had long been established by this period, there was still, as
Bruno Nettl notes in his 1964 monograph \emph{Theory and Method in
Ethnomusicology}, considerable fluidity of perspectives regarding
ethnomusicological methods \emph{per se} (12). Notable among the
theoretical paradigms discussed by Nettl is the concept of
bi-musicality, popularized and developed by Mantle Hood of UCLA in
numerous sources, but perhaps most directly ``The Challenge of
Bi-Musicality'' in 1960 (1964: 22). Though Mantle Hood's intention in
this elaboration of the bi-musicality concept is to comment upon the
practicalities and difficulties of developing expressive expertise in
multiple codified musical traditions (i.e.~those with particular
traditions of master-student teaching etc.), he speaks on multiple
counts about the distinct relationship between knowledge and performance
the theory engenders. The first of such statements is his declaration
that the ``crowning achievement in the study of Oriental music is
fluency in the art of improvisation,'' a state which ultimately requires
an assimilation of ``the whole tradition'' on the part of the student
(1960: 58). Leaving aside Hood's obviously dated language and deeply
reductionist view of Asian musical traditions, his preoccupation here is
clearly homologous to Adler's insistence on the circular relationship of
science and art, particularly as the two are realized in Adler's notion
of ``practical pedagogics'' (1983: 16). In other words, Hood here
gestures to a inductive relation between the ``life of art'' and its
constituent science, with the necessary caveat that the terms of
``life'' and ``science'' must be broadened to include the cultural codes
of the particular expressive art to be studied. A similar perspective is
demonstrated in Hood's investigation into the question of ``how far a
Western musician can go along the road of Oriental music studies''
(1960: 58). Hood makes clear that the question is not really one of
limits, but of time, evincing a perspective of complete expressive
fluidity. Particularly illustrative, however is his note that ``If
{[}the musician's{]} desire is to comprehend a particular Oriental
musical expression so that his observations and analysis as a
musicologist do not prove to be embarrassing, he will have to persist in
practical studies until his basic musicianship is secure'' (1960: 58).
What is most striking about this passage is the fashion in which it
foregrounds observation and analysis in musical practice itself; for
Hood, the process of rendering music is itself fundamental to accurate
study. As Nettl notes, this development of ethnomusicological method
resulted in a novel alignment of ethnomusicology with music departments
in the United States, a connection which had languished in the early
years of the field's formation (1964: 22).

It is in this fashion that we can theorize a strong relation between
artistic performance and the knowledge generated by ethnomusicology as
an Adlerian science of art. Of course, this is only one face of the
epistemological coin, for this period also saw the rise of Alan
Merriam's distinctive style of anthropological analysis, epitomized by
his landmark text, \emph{The Anthropology of Music.} In marked contrast
to the methods of Hood, which emphasize an embodied and dialogic
experience, Merriam hones in on the exterior position of the social
scientist. Indeed, the boundaries between social science and the
humanities generally seems to be a sore-spot for Merriam, as evidenced
by early theoretical passages in \emph{The Anthropology of Music}:

\begin{quote}
We are faced with the inevitable conclusion that what the
ethnomusicologist seeks to create is his own bridge between the social
sciences and the humanities. He does so because he must be involved with
both; although he studies a product of the humanistic side of man's
existence, he must at the same time realize that the product is the
result of behavior which is shaped by the society and culture of the men
who produce it. The ethnomusicologist is, in effect, sciencing about
music\ldots He does not seek the aesthetic experience for himself as a
primary goal (though this may be a personal by-product of his studies),
but rather, he seeks to perceive the meaning of the aesthetic experience
of others from the standpoint of understanding human
behavior\ldots Ethnomusicology endeavors to \emph{communicate knowledge
about an artistic product,} the behavior employed in producing it, and
the emotions and ideation of the artist involved in it (1960: 25).
\end{quote}

What this passage reveals is the primacy of the active verb
``sciencing'' as a process which is, at its foundation, a communication
of the truth attained by empirical investigation. It indicates that the
only method available in this project is an analytic which is disjunct
from the scholar's own active process of music-making and aesthetic
contemplation. Indeed, this emphasis on the communicative is articulated
most clearly a few pages earlier, in which Merriam states that
ethnomusicology is distinct from artistic practice insofar as ``it does
not seek to communicate emotion or feeling, but rather knowledge''
(1964: 19). The incommensurability of art and science's semantic object,
feeling and knowledge respectively, is paramount to Merriam's
distinction. Thus, for Merriam there is a single position which the
ethnomusicologist inhabits as a researcher, one which probes the
functional aspects of musical behavior and expresses results through
methods which are verifiably true; ``science'' rather than art-making,
or, in Adler's terms, ``the Discovery of the True'' has triumphed over
``the Advancement of the Beautiful''.

There can be no doubt that such strong statements by Hood and Merriam
set themselves up to be viewed in dialectical opposition, and indeed,
the tension between these perspectives has been a productive one within
the field. However, what the two outlooks share at their root is an
emphasis on the necessity of epistemic clarity. Hood's innovation was to
suggest how musical expertise impacts the process of observation which
forms the ground of inquiry. Where Merriam abstracts, noting that
ethnomusicology ``is both a field and a lab discipline'' and that it
``aims to approximate the methods of science,'' Hood opens the
possibility of observational fallacy, and the deeply subjective nature
of experiment (Merriam 1964: 37). Thus we may understand this early
instantiation of ethnomusicology as a discipline which is characterized
by its variable tolerance to performance as a research-method and the
emotional, aesthetic nature of communication which characterizes it.
However, this antinomy between the performance-oriented scholarship of
Hood and the empirical, functional school of Merriam was entirely
undercut by the development of the so-called ``Writing Culture'' debates
of the 1980's in cultural anthropology, destabilizing the notion of
communication which was so central to both the ``social sciences'' and
the project of ethnography as a whole. Adopting what has variously been
described as a textualist, post-structural, reflexive, or post-modern
stance, the co-authors of \emph{Writing Culture: the Poetics and
Politics of Ethnography} troubled the relationship of ethnographic
writing to what it sought to describe by highlighting its position
\emph{as writing,} which is always accompanied by the confines of genre,
aesthetics, and partiality (Zenker 2014). It is in this sense that James
Clifford, in his introduction to the volume, declares that all
ethnographies are \emph{fictions} in the sense that they are objects
with their own histories and systems of production -their claims to
truth in the scientific, empirical sense are entirely suspect (1986: 6).
These perturbations cut out the heart of any anthropologically-oriented
positivism; if ethnomusicology was to follow contemporary analyses of
culture it would, inevitably, have to confront its own position as a
\emph{writing} of music which was bereft of any claims on knowledge
distinct from its object of study.

While the impact of the writing culture debates on the use of
performance practice in ethnomusicology cannot be treated with any
finitude here, there are two examples which I find particularly
pertinent in demonstrating novel relationship of ethnographic writing to
performance initiated by these debates. The first text is Stephen
Slawek's 1994 elaboration of the guru-sisya relationship ``The Study of
Performance Practice as a Research Method: A South Asian Example,''
based on his own position as a senior disciple of Pandit Ravi Shankar, a
virtuoso performer of the sitar. A key feature of Slawek's approach is
the complex position that the results of fieldwork, variously termed
``data,'' ``collectanea,'' etc., can inhabit. While terms with a strong
postivist legacy like ``data'' clearly indicate a perspective which has
yet to fully embrace the fictive nature of ethnographic writing, Slawek
notes that a key feature of the guru-sisya relationship is an ethical
imperative to conceal certain details of musical instruction (1994:
9-10, 16). Indeed, Slawek does not hide the complicated position that
such cultural norms force him into:

\begin{quote}
The insights I have gained into Indian musical culture by becoming a
sisya transcend the complex vocabulary of music. I now have a clear
understanding of the real life of the tradition\ldots When certain
aspects of playing technique or particular elements of a repertory are
designated as ``classified'', and I am forbidden by Pandit Ravi Shankar
to include them in my research data, I cannot help but gain a better
understanding of the value of this knowledge to him. Yet, it is vexing
to me as a scholar, and certainly is the most extreme example of the
drawbacks inherent in attempting to combine the role of sisya with that
of researcher. (16)
\end{quote}

Clearly such a complex relationship to data, one in which experience may
or may not be distilled into publishable theory but may quite seriously
impact the researcher's relationship to further investigations, is
simultaneously the apogee of Mantle Hood's notion of ``assimilation''
and a devastating blow to the epistemic paradigm which undergirds it.
Finding himself at an unique impasse, squarely positioned in ``the
crisis of representation'' brought into such contention by Writing
Culture debates (Zenker 2014), Slawek makes a gamble on the actual
object to which he has gained access through the guru-sisya
relationship: musical performance itself. His writing on this subject
explicitly references \emph{Writing Culture}, noting that the in some
sense ethnomusicologists with a strong relationship to performance have
been doing the work of destabilizing ethnography ``without getting
credit'' (1994: 22). More important than credit, however, is Slawek's
assertion of the potential of performance itself, the successful
execution of which may ``equal a written document in intellectual
engagement and most probably will surpass a written statement in the
intensity of its emotive affect'' (1994: 22).

The second document which underlines the distinct contributions of
\emph{Writing Culture} is Jeff Titon's elevation of the question of
fieldwork to the grounds of ontology in the seminal anthology
\emph{Shadows in the Field.} In discussing the questions of what can be
known about music and how such knowledge might be created, Titon pursues
an alternative tack, suggesting that such knowledge is best solved
through an exploration of musical being. Key to Titon's argument is the
conception of a ``special ontology'' of musical being, which he
elaborates in the following passage:

\begin{quote}
Another way of saying this is that I ground musical knowledge in the
practice of music, not in the practice of science, or linguistics, or
introspective analysis. In my paradigm case of musical
being-in-the-world I am bound up socially with others making music and
when that music is presented fully to my consciousness it is the music
of the whole group, not simply ``my'' music, although at peak moments I
feel as if it is all coming through me. (32)
\end{quote}

This emphasis on the subjective account of musical knowledge, produced
through a collaborative, but \emph{perceptually subjective} experience
is anathema to the inductive knowledge and empirical character which
dominated ethnomusicological discourse in the '60s. Much like Slawek's
attestation of the affective and scholarly potential of performance,
Titon suggests here that musicality is itself the site of access to
musical knowledge; where for Slawek the project of public scholarship
was limited by ethical and cultural demands, for Titon the musical
subject \emph{is itself} one which has its foremost presence in the
experiential. This phenomenological account of musical experience is
taken up in the same volume by Timothy Rice, whose acquisition of
Bulgarian \emph{gaida} ornamentation exceeded the capacity of his
teacher to explain, and was produced through intense examination of
recordings and personal instruction. In Rice's case, the dialogic
relationship between study in ``the field'' and outside of it produces a
kind of knowledge which is neither emic nor etic, but one which attests
its validity in experience and in verification through others
(2008:51-52).

With the sanctity of ethnographic writing thus perturbed and the
fundamental epistemological structure of ethnomusicology shifted to
include performance as a scholarly document, it seems necessary to pose
the question: in what sense can the distinction between the artist's
``communication of feeling'' and the scientist's ``communication of
knowledge'' be meaningfully maintained? And indeed, if such a
distinction is collapsed, what could distinguish the nature of the
resultant scholarship from artistic practice itself? Have we escaped
Adler's paradigm of the true and the beautiful, or have we simply
reproduced them in the form of a single entity? In discussing these
questions, I will outline the development of \emph{sensory ethnography}
in cultural anthropology and its use of notions like affect, ``thick
depiction,'' and the sensory, to justify artistic forays into the
presentation of ethnographic research. In so doing, I will discuss the
development of arts-based research in the social sciences, often termed
``research-creation'' in Canadian academic contexts (Chapman and Sawchuk
2012), and discuss the applicability of Donna Haraway's notion of
``situated knowledges'' to mediate contesting epistemes.

\hypertarget{tags}{%
\section{Tags}\label{tags}}

\texttt{\#in-progress}

\hypertarget{refs}{}
\begin{cslreferences}
\leavevmode\hypertarget{ref-zenker_writing_2014}{}%
Zenker, Olaf. 2014. ``Writing Culture.'' \emph{Oxford Bibliographies
Online}.
\url{https://www.oxfordbibliographies.com/view/document/obo-9780199766567/obo-9780199766567-0030.xml}.
\end{cslreferences}
